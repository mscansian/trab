\documentclass[12pt]{article}
%\nonstopmode %Debug: Evita erros de packages nao inclusos

%Codificacao do arquivo e acentos
\usepackage[T1]{fontenc}
\usepackage[utf8]{inputenc}
\usepackage[brazilian]{babel}

%Fonte e formatacao de paragrafos
\usepackage{times}
\usepackage{indentfirst}
\usepackage{setspace}

%Pacote para carregar imagens
\usepackage{graphicx}

%Margens e formatacao de pagina
\usepackage{geometry}
\geometry{a4paper, total={210mm,297mm}, left=30mm, right=20mm, top=30mm, bottom=20mm}

\begin{document}

	%Capa do relatorio
	\begin{titlepage}
		\begin{center}
			{\fontsize{16pt}{\baselineskip}\fontfamily{\familydefault}\selectfont MATHEUS GIARETTA CANSIAN}\\[6cm]
			{\fontsize{18pt}{\baselineskip}\fontfamily{\familydefault}\selectfont \bf RELATÓRIO DE ESTÁGIO CURRICULAR}\\[5.5cm]
		\end{center}
		
		{
			\fontsize{14pt}{\baselineskip} \fontfamily{\familydefault} \selectfont
			
			\hspace{.45\textwidth} \begin{minipage}{.5\textwidth}
				\noindent 
				Relatório apresentado ao Curso de Engenharia Mecânica do Centro de Ciências Tecnológicas, 
				da Universidade do Estado de Santa Catarina, 
				como requisito parcial para a obtenção do grau de Bacharel em Engenharia Mecânica\\[0.6cm]
				Orientador: Fernando Hummel Lafratta\\[0.1cm]
				Supervisor: Michael Thamm
			\end{minipage}
		}\\[2.5cm]
		
		
		\begin{center}
		{
			\fontsize{14pt}{\baselineskip} \fontfamily{\familydefault} \selectfont
			Joinville, SC\\[0.2cm]
			2014
		}
		\end{center}
	\end{titlepage}

%Espacamento 1.5
\onehalfspacing

%Sumario
\tableofcontents
\pagebreak

%Lista de figuras
\listoffigures
\pagebreak

\section{Resumo}
\pagebreak

\section{Abstract}
\pagebreak


\section{Introdução}
	Este relatório é referente ao estágio obrigatório realizado pelo aluno Matheus Giaretta Cansian, do curso de Bacharelado em Engenharia Mecânica, da instituição UDESC campus Joinville.
	O período no qual o estagiário atuou foi de 11 de abril de 2014 a 31 de dezembro de 2014, na empresa Husqvarna Construction Products, localizada na cidade de Niederstotzingen, Alemanha.
\pagebreak

\section{Apresentação da concedente}

A Husqvarna é líder global em equipamentos para manejo de florestas, gramados e cuidado com o jardim. O grupo tambem é líder na Europa em produtos para irrigação residencial e um dos líderes mundiais em equipamentos para corte e ferramentas de diamante para a indústria da construção civil. As soluções desenvolvidas pelo grupo chegam ao mercado principalmente através de revendedores, tanto para uso doméstico quanto profissional. A Husqvarna está presente em mais de 100 paises, com 14 mil funcionários e um faturamento anual de 4 bilhões do dólares.

\subsection{Marcas}
	Para atingir grupos distintos de consumidores a empresa não só utiliza a marca Husqvarna, como também outras, sendo Husqvarna, Gardena, McCulloch e Diamant Boart as principais marcas do Grupo.

\subsubsection{Husqvarna}

\begin{figure}[h!]
	\centering
	\includegraphics[width=0.4\textwidth]{img/logo-husqvarna.png}
	\caption{Husqvarna}
\end{figure}

	Husqvarna é, há muitos anos, uma marca premium e forte em todo o mundo, representando a liderança tecnológica, desempenho profissional, alta qualidade e foco no usuário. A marca Husqvarna é responsável por aproximadamente 50\% das vendas do Grupo.

\subsubsection{Gardena}

\begin{figure}[h!]
	\centering
	\includegraphics[width=0.4\textwidth]{img/logo-gardena.png}
	\caption{Gardena}
\end{figure}

	Gardena é a marca premium do canal varejo, líder na Europa em produtos para irrigação e ferramentas de jardim para o uso doméstico. A linha também inclui produtos movidos a bateria e representa aproximadamente 10\% das vendas do Grupo.

\subsubsection{McCulloch e Diamant Boart}

\begin{figure}[h!]
	\centering
	\includegraphics[width=0.4\textwidth]{img/logo-mcdb.png}
	\caption{McCulloch e Diamant Boart}
\end{figure}

	McCulloch é uma marca premium global, incluindo produtos para manejo de florestas e jardins para consumidores exigentes do canal de varejo. 
	Diamant Boart é reconhecida como a marca líder global na indústria de pedras. A oferta de produtos inclui uma linha completa de ferramentas diamantadas para o processamento de pedra natural.	

\subsection{Visao}
	Vislumbramos um mundo onde as pessoas possam desfrutar de jardins, parques e florestas bem cuidados e experimentar estradas e edifícios refinados.

\subsection{Misao}
	Fornecemos soluções e produtos inovadores e de qualidade para tornar mais fácil o cuidado de jardins, parques e florestas, bem como as atividades do setor de construção, para profissionais e consumidores ao redor do mundo.

%Construction é responsavel por 10\% do faturamento do grupo
	
\subsection{Historia}
	Em 1620, foi fundada a empresa “Jönköping Rifle Factory”, por decreto do rei da Suécia e durante os primeiros anos essa fábrica produziu cerca de 1.500 tubos de mosquete anualmente. A assinatura do produto inspirou o logotipo clássico “mira de mosquete” que, apesar de atualizado, é usado ainda hoje.
Quando a Suécia começou a aumentar seu exército em 1689, nasceu oficialmente a fábrica da Husqvarna para perfuração e moagem de tubos de mosquete, a 7 km de Jönköping nas cachoeiras Huskvarna (escrita antiga de Husqvarna) - um lugar onde agora se localiza o moderno complexo da fábrica da Husqvarna.

	O contrato de produção de rifles Husqvarna para a Coroa chegou ao fim e a empresa começou a procurar maneiras de diversificar. Isto se tornou o início de um período muito inovador e ambicioso, que resultou em uma ampla gama de novos produtos, tais como: máquinas de costura (1872), armas de caça (1877), fogões de madeira (1884), máquinas de moer (1890), a primeira máquina de escrever sueco (1895), bicicletas (1896), motocicletas (1903) e fogões a gás (1912).

	Em 1918 a Husqvarna adquiriu “Norrahammars Bruk”, adicionando dois novos produtos ao seu portfólio: caldeiras e cortadores de grama manuais. Em 1919, a empresa começou a fabricar seus próprios motores.

	Em 1972 o nome da empresa foi oficialmente abreviado para simplesmente “Husqvarna”, que foi implementado juntamente com o logotipo atual.

\subsection{Principais produtos}
Os produtos na área de construção se dividem em categorias:

\subsection{Mercado de construção civil na Alemanha}
	O mercado de construção civil na Alemanha é marcado pelo alto custo da mão de obra. Por este motivo, máquinas que agilizam a construção são muito funcionais e demandadas. No portfolio da empresa existem máquinas como uma serra de parede que custa mais de 150 mil reais. Esse custo se justifica, pois um equipamento desse porte agiliza o trabalho e faz com que uma obra seja feita em tempo padrão, com mão de obra menor. Existem outros produtos como uma furadeira automática que aumenta a produtividade em 2 vezes: o operador monta a máquina e realiza um furo e no tempo em que essa está opearando, ele pode montar uma segunda máquina. Quando a segunda furandeira esta pronta, a primeira já terá terminado o furo.

Outra peculiaridade das contrucões alemãs é que os premoldados são muito utilizados. Isso diminui o tempo necessário para executar uma obra.

\subsection{Tipos de clientes}

%(O QUE SIGNIFICA A SIGLA OTS?)
%On-the-shelf

	O mercado da Husqvarna é composto por dois perfis de clientes com características distintas. O primeiro deles é chamado internamente de "OTS" e corresponde aos revendedores de máquinas, que são responsáveis por distribuir essas máquinas aos profissionais de pequeno porte e outros consumidores interessados em produtos de construção. Para este grupo, a Husqvarna fornece máquinas mais genéricas e menos profissionais, sendo as serras circulares, as serras de mesa e as furadeiras mais simples os principais produtos consumidos. A decisão de compra é caracterizada por maior foco no aspecto marketing e menos em aspectos técnicos.

\begin{figure}[h!]
	\centering
	\includegraphics[width=0.9\textwidth]{img/k760-vs-k1260rail.png}
	\caption{K760 e K1260 RAIL}
	\label{fig:k700vsk1260}
\end{figure}

\begin{figure}[h!]
	\centering
	\includegraphics[width=0.9\textwidth]{img/clientes.png}
	\caption{Tipos de clientes}
	\label{fig:tipo-clientes}
\end{figure}

	O segundo grupo é conhecido como "heavy users" e corresponde aos prestadores de serviço de médio e grande porte. Esse público necessita de máquinas mais profissionais e robustas e os principais produtos consumidos são as serras mais pesadas, serras de chão, robôs de demolição e serras de parede. A decisão de compra, nesse caso, é mais focada nos aspectos técnicos do produto (custo de manutenção, potência, eficiência) e isso exige da Husqvarna um portfólio com soluções extremamente especializadas, que realizam determinados serviços com a maior eficiência possível. Os heavy users realizam serviços diversos de construção e em geral, utilizam o maquiário no dia-a-dia. A sofisticação dessa linha de produtos requer suporte e manutenção mais cuidadosos, pois alguns produtos tendem a ter mais de 400 horas de uso ao ano. 
	
	Na figura \ref{fig:k700vsk1260}, pode-se visualizar a diferença entre uma máquina genérica (à esquerda) e uma máquina especializada em cortar de trilhos de trem. A máquina especializada, apesar de executar apenas um tipo de serviço, o faz com melhor eficiência. Na figura \ref{fig:tipo-clientes} pode-se visualizar as pricipais características de cada grupo de clientes.
	


\subsection{Estrutura administrativa}

\begin{figure}[h!]
	\centering
	\includegraphics[width=0.9\textwidth]{img/organograma.png}
	\caption{Organograma da empresa}
	\label{organograma}
\end{figure}

\subsection{Instalações}
	
	Na Alemanha a empresa não possui plantas de produção, porém existem dois depósitos e um escritório de vendas. O escritório - responsável por toda a Alemanha - e um dos depóstitos estão localizados na cidade de Niederstotzingen, em Baden-Wurtemberg, à cerca de 100 km de Stuttgart. O segundo depósito fica na cidade de Müllheim, próximo à fronteira com a França.

	Diante da ausência de planta de produçao no país, o número de funcionários na Alemanha é baixo. São menos de 50, que fazem todas as atividades de vendas, despacho, compras e afins.

\section{Teoria}

\subsection{Preco, Custo e Valor}
Existem alguns conceitos econômicos envolvidos em operações de venda que é imortante compreender, porém são normalmente confundidos entre si. São eles:

\subsubsection{Preco}
Preço é a quantidade de pagamento ou compensação dado por uma parte à outra em troca de produtos ou serviços. Na economia moderna, o preço é normalmente expressado em relação a uma moeda corrente. Juntamente com a definição de preço existe uma outra conhecida como Preço de Venda. Este nada mais é que o valor que uma parte pede à outra em troca de mercadorias ou serviços. O Preço de Venda pode ser diferente do Preço da Transação se, por exemplo, o vendedor der algum desconto.

\subsubsection{Custo}
Custo corresponde à quantidade de dinheiro utilizado para produzir algo. Os custos podem ser dividdos em fixos e variáveis. Custos variáveis são todos aqueles que estão diretamente ligados à produção de algo. Quanto maior a quantidade de itens prodruzidos, maior o custo variável. Por outro lado, os custos fixos são custos necessários para a produção de algo, mas que não variam com a quantidade de itens produzidos. Um exemplo de custo fixo é o aluguel de um galpão de fábrica.

\subsubsection{Valor}
O valor é muitas vezes confundido com o preço, pois normalmente é expressado em unidade de moeda corrente, porém trata-se de um conceito econômico que é definido por alguns fatores como: usabilidade intrínseca ao produto, a demanda que o mercado tem por este produto, bem como a sua oferta. É possível resumir o conceito de valor respondendo à seguinte pergunta: "Qual é o preço máximo que um cliente pagaria por este produto?".

Dentro do marketing existe um outro conceito chamado de Valor Percebido. Ele é, basicamente, a divisão entre o Valor de um produto e o seu Preço. O Valor Percebido é o que rege a ação de compra de um determinado consumidor. Quanto maior for o valor percebido, mais provável é que o consumidor efetue a compra de um produto ou serviço. Vale lembrar que o Valor Percebido não é simplesmente o valor econômico, mas sim algo mais subjetivo, que leva em conta questões como popularidade, grife, sazonalidade, etc.

Um exemplo simples da subjetividade do Valor é um pacote de café brasileiro vendido no Brasil e na Europa. Os clientes europeus veem um valor maior no produto (por ter vindo do Brasil) do que os clientes brasileiros. Se o pacote de café fosse europeu, essa relação de valor provavelmente se inverteria.

\subsection{Financas Corporativas}

	Para uma empresa o indicador mais confiável do seu sucesso presente é o lucro líquido. No entanto, existem muito outros indicadores intermediários que também são muito importantes. Todos esses indicadores possuem uma regra geral definida acadêmicamente, entretanto a sua aplicação no dia-a-dia da empresa nem sempre segue a regra a risca. Na maior parte dos casos, alguma simplificações são realizadas. Elas são O objetivo delas é a economia de tempo, com uma perda pouco significativa no resultado final.

%(MOZINHO, ESSE PARÁGRAFO DE CIMA ESTÁ MUITO LINGUIÇADO. NÃO FICOU CLARO O QUE VC QUIS DIZER.)
%melhorou?

\begin{figure}[h!]
	\centering
	\includegraphics[width=0.9\textwidth]{img/finance.png}
	\caption{Representacao esquematica do calculo de lucro liquido}
	\label{fig:lucro}
\end{figure}

A figura \ref{fig:lucro} mostra uma das representações mais simples utilizadas para o cálculo do lucro líquido através de deduções da receita obtida pela empresa. Nos itens abaixo serão discutidos alguns dos indicadores intermediários, seus cálculos e utilização. 

\subsubsection{Receita}
Receita é a entrada monetária que ocorre dentro de uma empresa, normalmente relativa à vendas de mercadorias ou serviços. As receitas podem ser brutas ou líquidas e operacionais ou não-operacionais. A receita bruta é aquela que corresponde ao valor negociado na aquisição de um produto e é a utilizada para cálculo de impostos sobre ascvendas. A receita líquida então é a receita bruta descontada de devoluções, impostos diretos sobre o valor de venda e abatimentos.
A receita operacional é toda aquela proveniente da atividade principal de uma empresa, enquanto a receita não-operacional é resultado de atividades não principais da empresa. Um exemplo de receita não-operacional são aplicações financeiras.

\subsubsection{Desconto e Abatimento}
Desconto e abatimento são reduções no preço de um produto realizadas para determinados clientes ou grupo de clientes. A diferença básica entre desconto e abatimento é que o primeiro refere-se a uma redução realizada antes da emissão da nota fiscal, enquanto o segundo é realizado depois. Comumente os descontos são utilizados por empresas para incentivar um cliente a comprar determinado produto ou a não comprar um produto similar do concorrente. Abatimentos são utilizados para incentivar um cliente a comprar quantidades maiores de determinados produtos. As compras podem ser realizadas em datas diferentes e o cliente recebe um abatimento no valor da compra quando atinge uma determinada meta de volume.

\subsubsection{Custos Variaveis}
Custos variáveis são todos aqueles que têm relação direta com a quantidade produzida. Para uma indústria, os custos variáveis podem ser: matéria prima, mão de obra direta, energia elétrica direta, entre outros. Para o caso de um escritorio de vendas, os custos variáveis são o preço interno e o custo de transporte da fábrica até o depósito.

\subsubsection{Lucro Bruto}
Lucro Bruto corresponde à receita líquida menos os custo variáveis. Ele revela o quanto o produto gera de lucro, sem levar em consideração a estrutura administrativa e seus respectivos custos fixos.

\subsubsection{Custos Fixos}
Custos fixos sao todos os que nao são proprocionais ao volume produzido. Isso inclui o salário do setor administrativo, aluguel, energia utilizada em atividades não produtivas, etc.

%MOZI, ACIMA, VOLUME VENDIDO OU PRODUZIDO?
%produzido :D
%beleza :p

\subsubsection{EBIT}
EBIT é a sigla em inglês para Earnings Before Interest and Taxes (Lucro antes de Juros e Imposto de Renda). Ele é correspondente ao lucro total obtido pela a empresa sem levar em conta a política de distribuição de lucros, juros de empréstimos anterioes, bem como impostos sobre o lucro. O EBIT tem duas finalidades: a primeira é enquanto indicador de como a empresa esta perfomando no presente - como este valor não leva em conta o pagamento de juros, o lucro da empresa do período não é punido por empréstimos que a empresa fez no passado; a segunda função é ter uma relação entre duas empresas distintas - como o EBIT desconsidera o pagamento de impostos, ele não está punindo uma empresa que encontra-se em um pais no qual os impostos são maiores. O EBIT também não considera a distribuição de lucros, ou seja, não pune uma empresa que distribui mais lucros para os seus acionistas.

\subsubsection{Impostos}
Impostos são a última dedução realizada antes do lucro líquido. O imposto dedudzido nesta etapa corresponde ao imposto sobre o lucro obtido.

\subsubsection{Lucro Liquido}
Lucro líquido é o capital que sobra para a empresa após cumprir todas as suas obrigacões fiscais e legais.

%PAREI AQUIIIIIIIIIII!!!!!
% <3
%Pipi... e lá vamos nós...

\subsection{Efeitos Financeiros}
Efeitos financeiros são os impactos que ocasionam uma diferença financeira entre dois períodos. A maior utilização desse conceito é para verificação de quais podem ter sido as causas que trouxeram uma diferença de lucro, rentabilidade ou custo num de terminado período. Não existe classificação fixa para quais são os efeitos financeiros. Por regra, os mais utilizados para o cálculo de diferença de lucro são: Volume, Preço, Custo e Mix.

\begin{figure}[h!]
	\centering
	\includegraphics[width=0.6\textwidth]{img/effects.png}
	\caption{Representacao esquematica dos efeitos de volume e preco}
	\label{fig:effects}
\end{figure}

Na figura \ref{fig:effects} pode-se ver uma representação simples dos efeitos de volume e preço na rentabilidade entre dois anos. Cada retângulo corresponde a um produto vendido e a sua área a rentabilidade total. No primeiro ano foram vendidos quatro produtos. No ano seguinte, foram vendidos sete produtos, com um leve aumento de preço. A área em amarelo corresponde ao faturamento referente ao ano anterior; em verde, devido ao aumento de preço e em vermelho, devido ao aumento de volume.

\subsubsection{Efeito Volume}
Corresponde ao lucro adicional, ocasionado pelo aumento de volume de vendas.

\subsubsection{Efeito Preco}
\subsubsection{Efeito Custo}
\subsubsection{Efeito Mix}

\section{Projetos Realizados}

	O estágio foi realizado no escritório de vendas da Alemanha, sendo as principais atribuições do estagiário o desenvolvimento de projetos para aumentar indicadores de vendas e a prestação de apoio nas atividades do dia-a-dia.  Durante o período, dois projetos foram de grande importância e ajudaram a desenvolver áreas da empresa que estavam aquém de seu potencial. O primeiro deles trata-se de uma nova metodologia para precificação de peças de reposiço, enquanto o segundo foi direcionado ao desenvolvimento de um programa de manutenção para um dos produtos fabricados pela Husqvarna. Devido ao tempo do estágio, o segundo projeto limitou-se a apenas uma máquina, mas o modelo de negócio escolhido pode ser utilizado para os demais produtos.

\subsection{Política de precificação para peças de reposição}

	Os equipamentos da Husqvarna para a construção civil são focados no público profissional. Isso significa que as máquinas devem ser desenvolvidas para uma alto número de horas de uso. Por esse motivo, falhas são frequentes e a disponibilidade e preço das peças de reposição são dois dos principais motivadores que fazem as empresas comprar os produtos do Grupo. O objetivo deste projeto foi analisar a precificação que a Husqvarna e seus concorrentes fazem para as peças de reposição, bem como a maneira como o cliente avalia o "valor" de uma peça. A partir disso, foi montada uma estratégia de preços visando aumento do faturamento da empresa e da satisfação do cliente.

	Na maioria da empresas, a conta que se realiza para o cálculo de preço é, basicamente, utilizar o custo de determinado produto e adicionar uma margem esperada. O problema dessa abordagem é que, não levando em conta o preço do mercado, pode-se estar vendendo mais caro do que o cliente esperava pagar (perdendo assim volume) ou vendendo mais barato do que o cliente pagaria (perdendo assim lucro). A maneira ideal de se obter o preço de um produto é fazer a análise inversa: Qual é o preço máximo que o meu cliente estaria disposto a pagar por este produto? Atráves da resposta à essa questão, é possível descontar do preço uma margem interessante e lucro para a empresa e, com isso, estabelecer o custo máximo viável para que esse produto seja considerado rentável.

	A Husqvarna, por ser uma empresa global, tem uma dificuldade especial: a diferença de preços entre países. A Europa é um continente no qual os laços comerciais entre os países são muito fortes e é facil para uma empresa analisar o preço de determinado produto no exterior. Diante disso, os preços da Husqvarna devem fazer sentido não só para a Alemanha, mas devem estar alinhados com o que os outros países cobram (França, Países Baixos, Áustria, etc). Via de regra, os preços em determinado país são mais altos, mas isso aplica-se a todo o portfólio de produtos e não apenas a um produto, isoladamente.

	Outro fator importante nessa análise é o fato de que muitos clientes heavy users possuem várias máquinas diferentes. Isso permite que eles façam comparações entre máquinas. Um exemplo disso seria um cliente que possui uma serra grande e uma pequena: ele pode comparar os preços das peças de reposição e aceitar que o filtro de ar para uma máquina grande é mais caro que um para máquina pequena, porém julgar que o contrário não é coerente. Neste caso, o cliente pode entender que o filtro para a máquina pequena está muito caro.

	O primeiro passo desse projeto foi levantar esses problemas e relizar um Brainstorming de soluções. Sendo a Husqvarna uma empresa grande, o problema torna-se de difícil resolução e é esperado que nem todos os pontos para todos os produtos possam ser solucionaos. A ideia do projeto, entretanto, é ao menos amenizar essas dificuldades.

\begin{figure}[h!]
	\centering
	\includegraphics[width=0.9\textwidth]{img/pricing.png}
	\caption{Objetivos do projeto de precificacao}
	\label{pricing}
\end{figure}

	Historicamente, os preços das peças de reposição eram calculados com base no custo. As margens de cada filial eram previamentes definidas e somando-as ao custo, obtinha-se o preço final. O principal ponto de atenção desse "approach" é que ele não leva em conta o preço de mercado. Simplesmente adicionando uma margem ao custo, não existe qualquer garantia de que o produto vendido tem preço competitivo. Além do exposto, essa visão não contribui para a redução de custos da empresa, tendo em vista que ela não proporciona a definição de qual custo "alvo" deve ser buscado.

	O primeiro passo para corrigir esse problema foi trabalhar em conjunto com a matriz, obtendo preços internacionais para cada produto. Como primeiro balisador, obteve-se a média internacional de preços para cada produto. Observando esses números, notou-se que eles apresentavam alguns problemas. 

	É conhecido que o custo de fabricação de um produto depende muito da quantidade produzida. Quanto maior a quantidade, menor o custo. O fato é que as máquinas mais vendidas pela Husqvarna são as máquinas menores (pois atendem tanto ao público de uso doméstico quanto o profissional) e por isso o custo de suas respectivas peças de reposição são menores do que as de máquinas maiores. Para o cliente final, a estratégia de produção da Husqvarna, ou o seu portfólio de clientes, não tem nenhuma relação com a decisão de compra. O cliente utiliza mecanismos mentais simples para tomar essas decisões. Por exemplo, para o cliente, um motor de uma máquina menor deve ser necessáriamente mais barato do que o de uma máquina maior. Entertanto, devido a produção da Husqvarna, nem sempre isso acontecia.
	
%MOZI, É ISSO MESMO? A HUSQVARNA VENDE MAIS MÁQUINAS MENORES DO QUE MAIORES? ME PARECEU QUE AS NOMEAÇÕES ESTÃO INVERTIDAS.

	Para contornar essa visão do cliente, novos critérios tiveram que ser definidos para o cáculo do preço. Foi criado um índice chamado de "value driver" (TODO: preciso traduzir isso) que é, basicamente, o parâmetro que o cliente enxerga no produto como valor. No caso de um motor por exemplo, utilizou-se a cilindrada. Quanto maior a cilindrada, maior o preço. No caso de parafusos, foi o peso. Quanto mais pesado e "heavy duty" um parafuso, mais caro ele era. A tabela XX mostra o value driver utilizado para alguns grupos de peças.

%COLOCAR TABELA AQUI!!!

	Através do value driver, foi possível melhorar a análise de preço. Os preços que antes eram baseados apenas em custo, agora não dependem mais da estratégia de producao da Husqvarna e fazem mais sentido para o cliente final. No entanto, isso não foi suficiente. Para que fosse obtido preço ainda melhor, foi necessário descobrir qual era o preço de mercado de cada produto.

	O preço de mercado corresponde a uma média de preços realizada por todos os players dentro de um mercado específico. Com o objetivo de obter essa informação, realizou-se uma pesquisa de mercado através da obtenção de lista de preços de revededores multi-marcas, pedidos de cotação enviados para revendedores de concorrentes e através do "input" do time de vendas. Utilizando os dados coletados, foi realizado ajuste dos preços já calculados.

	Outro fator importante sobre precificação é que ela não pode ser realizada em movimentos bruscos. Para isso, foi definidida uma taxa arbitrária de 3\% para a mudança de preço, entretanto, alguns artigos seguiram uma regra diferente.

	Para os artigos com preço muito baixo, como parafusos, uma mudança de 100\% corresponde a poucos euros, o que de certa forma, não afeta o consumidor final. Seguindo essa teoria, foi determinado que todos os artigos com o preço atual e alvo menor do que 10 euros seria modificado diretamente para o preço alvo.
	
%NÃO ENTENDI DIREITO O PARÁGRAFO ACIMA.

Outro fator importante é que o cliente só tem a possibilidade de comparar o preço histórico de partes que ele já comprou. Por isso, foi criada uma nova categoria de produtos que não tiveram vendas desde 2011. Esse produtos também foram modificados para o preço alvo.

Por último, foram detectados alguns preços que estavam significamente deconexos da realidade de mercado, provavelmente resultantes de erro de digitação no preço. Esses produtos também passaram por correção diretamente para o alvo.

\subsection{Serviço de manutenção de serras de parede}

\begin{figure}[h!]
	\centering
	\includegraphics[width=0.4\textwidth]{img/ws220_produto.png}
	\caption{Serra de parede modelo WS220}
	\label{fig:ws220_produto}
\end{figure}

	Para um empresa prestadora de serviços de construção na Alemanha, os gastos referentes à aquisição e manutenção das ferramentas de trabalho correspondem a um valor expressivo do custo total da empresa. Além disso, o ambiente agressivo e o uso constante ao qual as máquinas são expostas são motivos frequentes de falhas, até mesmo nas máquinas mais robustas. Para um empresa menor, que possui poucas máquinas, a falha de uma delas pode ter consequências devastadoras na prestação de serviços. Além dos custos de reparo, os dias em que a empresa fica impossibilitada de trabalhar gera grande impacto negativo no fluxo de caixa.

	O objetivo do projeto foi desenvolver um pacote de manutenção para transferir o risco de fluxo de caixa do cliente para a Husqvarna. Uma venda suficientemente grande de pacotes de manutenção é para a Husqvarna uma boa maneira de mitigar o risco de fluxo de caixa, tendo em vista que apesar das falhas serem imprevisíveis, é possível que a empresa calcule uma média de gastos anuais. Com um número suficientemente grande de máquinas, a probabilidade de as falhas ocorrerem em um mesmo mês é extremamente baixa. Para esse projeto específico, foi escolhida uma serra de parede modelo WS220. O motivo da escolhe deve-se ao fato que esse modelo é lançamento no mercado, e por isso é mais fácil apresentar a novidade aos clientes.

	Utilizando a metodologia do Balance Scorecards (BSC), o projeto foi divido em etapas, cada uma correspondendo a um fator crítico de sucesso. Os fatores do BSC podem ser visualizados na figura \ref{fig:bsc}. Para o quesito Financeiro foi definida a realização de uma análise de custos do produto; para o fator Cliente, foi realizada análise na questão de precificação do produto e serviços; na questão de Processos Internos foi desenvolvida toda a documentação relacionada ao pacote, de forma que a aplicação desse projeto fosse realizada de forma fácil e sem erros por parte da empresa; por último, na questão Aprendizagem e Crescimento, foi criada documentação do projeto visando que a replicação para outros produtos fosse relativamente fácil. Na figura \ref{fig:service} pode-se ver uma análise geral das questões a serem respondidas pelo projeto.

\begin{figure}[h!]
	\centering
	\includegraphics[width=0.4\textwidth]{img/bsc.png}
	\caption{Matriz BSC}
	\label{fig:bsc}
\end{figure}

\begin{figure}[h!]
	\centering
	\includegraphics[width=0.9\textwidth]{img/ws220.png}
	\caption{Objetivos do projeto de servico}
	\label{fig:service}
\end{figure}

\subsubsection{Análise de custos e precificação}

	Para o custo de manutenção preventiva, utilizou-se o manual de reparos do produto para efetuar lista de serviços de manutenção que devem ser realizados a cada intervalo de tempo. Com base nesses dados, foi calculado o custo desses serviços com base no preço interno. Além disso, o custo da mão-de-obra foi determinado pelo time de manutenção. Com base nos serviços que devem ser realizados, o time deu uma estimativa de horas para completar. Com base nessa estimativa e no custo de uma hora de trabalho foi estimado o custo de mão-de-obra.

\begin{figure}[h!]
	\centering
	\includegraphics[width=0.9\textwidth]{img/ws220-waterfall.png}
	\caption{Calculo de custos do projeto}
	\label{fig:ws-waterfall}
\end{figure}

	Em relação à análise de custos, primeiramente foram obtidos os dados sobre taxas de falhas para cada componente da máquina. Esses dados correspondem à média de produtos que apresentam falha a cada 100 horas de uso. Esses dados foram então extrapolados para um cenário de mil produtos e 500 horas de uso. O custo total de reposição das peças que apresentaram falha foi calculado com base no preço interno + frete. Esse custo corresponde ao de manutenção corretiva. A mão de obra foi calculada utilizando o custo das peças e a estimativa da manutenção preventiva. Na figura \ref{fig:ws-waterfall} pode-se visualizar os dados finais da análise de custo (os valores foram removidos do gráfico por serem dados confidenciais). O custo total de manutenção é a soma dos custos de manutenção preventiva, corretiva e transporte da máquina. Com base nesse custo foi possível adicionar a margem de lucro da empresa e obter o preço final para o cliente.
	
\subsubsection{Documentacao externa e interna}

	Seguindo a análise do BSC para os quesitos de Processos Internos e Aprendizagem e Crescimento, foram desenvolvidas documentacoes externas e internas.

\pagebreak

\section{Observacoes referentes as informacoes do relatorio}
	Informações como dados financeiros, estratégias de marketing e dados sobre produtos são consideradas sigilosas e portanto, foram omitidas neste relatório, respeitanto a política de confidencialdade da Husqvarna.
\pagebreak

\section{Considerações finais}
\pagebreak

\end{document}

%THE EEEEEND (SO FAR)
